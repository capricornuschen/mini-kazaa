\chapter{Connessione TCP: Il working thread}
Il TCP nacque nel 1970 come frutto del lavoro di un gruppo di ricerca del dipartimento di difesa statunitense. I suoi punti di forza sono l'alta affidabilità e robustezza.
Il protocollo TCP serve a creare degli stream socket, cioè una forma di canale di comunicazione che stabilisce una connessione stabile fra due stazioni, in modo che queste possano scambiarsi dei dati.

\section{Caratteristiche TCP}
Il servizio offerto da TCP è il trasporto di un flusso di byte bidirezionale tra due applicazioni in esecuzione su host differenti. Il protocollo permette alle due applicazioni di trasmettere contemporaneamente nelle due direzioni, quindi il servizio può essere considerato "Full Duplex" anche se non tutti i protocolli applicativi basati su TCP utilizzano questa possibilità.
Il flusso di byte viene frazionato in blocchi per la trasmissione dall'applicazione a TCP (che normalmente è implementato all'interno del sistema operativo), per la trasmissione all'interno di segmenti TCP, per la consegna all'applicazione che lo riceve, ma questa divisione in blocchi non è per forza la stessa nei diversi passaggi.
TCP è un protocollo orientato alla connessione, ovvero prima di poter trasmettere dati deve stabilire la comunicazione, negoziando una connessione tra mittente e destinatario, che viene esplicitamente chiusa quando non più necessaria. Esso quindi ha le funzionalità per creare, mantenere e chiudere una connessione.
TCP garantisce che i dati trasmessi, se giungono a destinazione, lo facciano in ordine e una volta sola. Questo è realizzato attraverso vari meccanismi di acknowledgment e di ritrasmissione su timeout.
TCP possiede funzionalità di controllo di flusso e di controllo della congestione sulla connessione, attraverso il meccanismo della finestra scorrevole. Questo permette di ottimizzare l'utilizzo della rete anche in caso di congestione.

\section{Divisione del file in parti.}\label{sec:download_tcp}
NON SO MOLTO COSA SCRIVERE.... 