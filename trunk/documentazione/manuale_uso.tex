\chapter{Manuale d'uso}
Affrontiamo ora la parte piu' pratica del progetto.
\section{Installazione}
Anzitutto e' importante sapere che per poter utilizzare del prodotto da noi creato e' necessaria almeno una rete LAN o una rete INTERNET.
Si pu� anche utilizzarlo offline utilizzando le backdoor del proprio computer pero' il progetto perde la sua utilita'.
Un computer della rete deve fornire il servizio di bootstrap Server ( inserire il riferimento alla pagina specifica ).
All'avvio del programma l' utente pu� decidere se essere un SuperNode o un OrdinaryNode. 
A seconda della scelta appariranno a video le corrispettive figure ed il programma � pronto a funzionare.

\section{Come funziona}
	\subsection{Vita da SuperNode}
Se abbiamo deciso di essere SuperNode noi vogliamo oltre che poter cercare e scaricare sulla rete fornire il nostro servizio alla comunica globale.
	\subsection{Vita da OrdinaryNode}
Se abbiamo decido di essere OrdinaryNode noi vogliamo usufruire solamente del servizio di ricerca e download che offre Minikazaa.
	\subsection{Cercare e scaricare un file}
Per cercare e scaricare un file, � necessario inserire nella casella vuota vicino a trova il titolo del file che interessa trovare nella rete.
La nostra casella di ricerca interpreta la parola chiave inserita della casella come una $ *esempio* $ . Questo significa che tutti i file che contengono nel titolo la parola esempio verranno restituiti come canditati per lo scaricamento.
Nella tabellina sotto il campo dove � stata inserita la parola ora appariranno i riferimenti ai file che sono presenti sulla rete, i quali possono essere facilmente scaricati cliccandovi due volte sopra.
Il download che parte automaticamente � visibile e controllabile nella sezione chiamata FILETRANSFERT.

	\subsection{Aggiungere un file nella lista dei file condivisi}
Come tutti i servizi di file sharing minikazaa permette di aggiornare la lista dei file che � possibile scaricare.

	\subsection{}
