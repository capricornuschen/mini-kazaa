\chapter{Bootstrap Server}

Il bootstrap server si avvia dal main situato all'interno del file \verb|BootstrapService.java|. Questo file esegue subito un blocco di tipo \verb|try{}catch| nel quale si trovano le seguenti istruzioni.

\begin{lstlisting}
try {
Registry registry = LocateRegistry.createRegistry(2008);
BootStrapGui g = new BootStrapGui();

g.setLocationRelativeTo(null);
            
g.setVisible(true);

BootStrapServer bss = new BootStrapServer(g);

BootStrapServerInterface stub = 
	(BootStrapServerInterface) 
	UnicastRemoteObject.exportObject(bss, 0);

SupernodeCallbacksImpl client_impl = new SupernodeCallbacksImpl
	(new SupernodeList(), new NodeConfig());
            
SupernodeCallbacksInterface client_stub = 
	(SupernodeCallbacksInterface) 
	UnicastRemoteObject.exportObject( client_impl,0);
            
System.out.println("Ready to bind.");
registry.bind("BootStrap", stub);
registry.bind("Callback", client_stub);
            

} 
catch (Exception e) {
System.err.println("BootStrapService: " + e);
}
\end{lstlisting}