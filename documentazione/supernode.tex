\chapter{Super Node}
Il package \verb|lpr.minikazaa.supernode| non contiene un gran numero di classi peculiari, ma attinge molto dal package per gli Ordinary Node.
Fornisce però l'interfaccia per le \emph{callbacks} RMI che viene utilizzata da tutto il client.
Di seguito mostriamo le caratteristiche principali del package \verb|supernode|.
\section{L'interfaccia per le callback}
L'interfaccia per le callback fonisce due metodi semplici ma fondamentali.
\begin{lstlisting}
public void notifyMeAdd(NodeInfo new_node)
	throws RemoteException;

public void notifyMeRemove(NodeInfo new_node)
	throws RemoteException;
\end{lstlisting}
Questi due metodi servono al client per mantenersi aggiornato sui SN presenti nella rete.
L'implementazione di questi due metodi è molto semplice poichè opera solo delle \verb|add| o \verb|remove| alla struttura \emph{SupernodeList}.
Non è pertanto interessante mostrarne il codice.

\section{Indicizzamento dei file degli ON collegati}
Come abbiamo già detto, gli Ordinary Node condividono i propri file attraverso il Super Node che scelgono come ``amico''. Un Super Node quindi ha bisogno di uno strumento per indicizzare le informazioni sui file condivisi che gli arrivano dalla rete.

Per questo motivo è stata predisposta una classe il cui obiettivo è tenere in memoria tutti i file condivisi dagli Ordinary Node.
\verb|SupernodeOnFileList.java| svolge proprio questo compito.
\`{E} composta da un unico attributo che viene inizializzato nel costruttore come segue.
\begin{lstlisting}
private ArrayList <OrdinarynodeFiles> file_list;

public SupernodeOnFileList(){
	this.file_list = new ArrayList();
}
\end{lstlisting}
Questa classe mette a disposizione anche diversi metodi utili a manipolare le informazioni che si trovano all'interno dell'\emph{ArrayList}.
\subsection{addNewOnFileList}
Un metodo che serve all'inserimento di nuove liste di files. Questo metodo si accorge se è già presente una lista di file per un determinato nodo e, nel caso, sostituisce la lista vecchia con la lista nuova.
Ciò semplifica la gestione delle liste riducendo le operazioni di add, update e remove a una sola operazione, e di conseguenza gli errori in cui si può incorrere.
Nel caso non esista già una lista per il nodo mittente, si aggiunge un nuovo campo all'\emph{ArrayList} \verb|file_list|.
Mostriamo quindi il frammento di codice che si occupa di queste operazioni.
\begin{lstlisting}
for(OrdinarynodeFiles o : this.file_list){
	if(o.getOwner().getId().
		equals(new_file_list.getOwner().getId())){

		//Abbiamo gia' informazioni per il
		//nodo owner
		o.resetList(new_file_list.getFileList());
		return;
	}
}

this.file_list.add(new_file_list);
\end{lstlisting}

% public synchronized ArrayList <OrdinarynodeFiles> searchFiles(String regex){
\subsection{searchFiles}
La funzione di ricerca dei file viene fatta come per la classe OrdinarynodeFiles, Sezione \ref{sec:on_searchFiles}.
La differenza in questo caso è che si lavora su liste di OrdinaryNodeFiles, quindi bisogna distinguere gli \emph{owner} di ogni lista di file.
Il codice è uguale a quello di Sezione \ref{sec:on_searchFiles} per cui non vale la pena riportarlo nuovamente.

\subsection{Altri metodi}
La classe prevede altri metodi per la rimozione delle liste o per verificare se un file è già presente all'interno della struttura dati, ma sono metodi molto semplici composti da poche righe, quindi li citiamo, ma non li mostriamo completamente.

\section{Smistamento delle query}\label{sec:smistamento_delle_query}
%SupernodeQueryList.java
Il compito principale di un Super Node, che poi è anche quello che lo differenzia principalemente da un Ordinary Node, è quello di smistare le \emph{query} nella rete, ovvero spargere le richieste nella rete.

Per fare questo però un Super Node non solo deve inviare le \emph{query} ai nodi che ritiene ``più vicini'', ma anche far ritornare le \emph{query} al mittente una volta ricevute le risposte.

Questo compito viene svolto dal \emph{TCPWorkingThread} sfruttando una struttura dati che si chiama \verb|SupernodeQueryList|.
Prima di tutto vediamo come questa struttura dati è costrutita.

Si compone di un solo attributo.
\begin{lstlisting}
private ArrayList <Query> query_list;
\end{lstlisting}
Questo attributo viene inizializzato nel costruttore a una lista vuota.
La classe mette anche a disposizione:
\begin{itemize}
 \item \textbf{getRelativeQuery}: data una query di risposta ricevuta in input restituisce la relativa che è stata ricevuta in precedenza;
 \item \textbf{generateQueryList}: prende in input una query di ricerca e una lista di node info e genera $n$ query per gli $n$ nodi contenuti nella lista passata come parametro;
\end{itemize}



\section{Il cuore di un Super Node}
\subsection{Engine}
Il motore di un Super Node è il task 
\begin{verbatim}
public class SupernodeEngine implements Runnable
\end{verbatim}
\verb|SupernodeEngine| ha il compito di inizializzare le variabili e gli oggetti che poi utilizzaranno i \emph{thread} dell'ON e di far partire tutti i vari task che controllano le varie interfacce.

Guardiamo ora, in un piccolo frammento di codice, le variabili e gli oggetti istanziati dall'\emph{engine} dell'Ordinary Node.
\begin{lstlisting}
NodeInfo my_infos = 
	new NodeInfo();

SupernodeList sn_list = 
	new SupernodeList();

SupernodeOnFileList on_files = 
	new SupernodeOnFileList();

OrdinarynodeFiles my_file_list = 
	FileUtil.loadMySharedFiles(my_infos);

OrdinarynodeQuestionsList found_list = 
	new OrdinarynodeQuestionsList();

OrdinarynodeDownloadMonitor dl_monitor = 
	new OrdinarynodeDownloadMonitor();


BootstrapRMIWrapper rmi_stub = 
	new BootstrapRMIWrapper();
\end{lstlisting}
Fa quindi partire i \emph{thread} che possiamo vedere nel seguente listato.
\begin{lstlisting}
//Inizializza la main GUI del supernode
MainGui main_gui = new MainGui(
		this.my_conf,
		my_file_list,
		found_list,
		sn_list,
		my_infos,
		dl_monitor,
		rmi_stub,
		null);
main_gui.setLocationRelativeTo(null);
main_gui.setVisible(true);

	
//Inizializza il servizio ping per ricevere i ping
NodePong pong = new NodePong(this.my_conf);
Thread ping_service = new Thread(pong);
ping_service.start();

//Inizializza RMI manager Thread.
SupernodeRMIManager sn_rmi = new SupernodeRMIManager(
		my_conf,
		sn_list,
		my_infos,
		rmi_stub);
Thread rmi_manager = new Thread(sn_rmi);
rmi_manager.start();

//Inizializza TCP requests manager
SupernodeTCPListener listener_tcp = new SupernodeTCPListener(
		this.my_conf,
		sn_list,
		on_files,
		my_file_list,
		dl_monitor,
		found_list);
Thread tcp_listen = new Thread(listener_tcp);
tcp_listen.start();
\end{lstlisting}
I \emph{thread} dei quali viene eseguito il comando \verb|.start()| sono quelli che poi andranno a gestire le varie interfacce che un client Mini-KaZaA, in particolare un Super Node, deve avere.

\subsection{SN e RMI}

\subsection{SN in ascolto su socket TCP}
Ogni client Mini-KaZaA deve stare costantemente in ascolto sul socket TPC poichè da esso giungono la maggior parte delle comunicazioni e un Super Node non fa eccezione.
Ogni client, pertanto, dedica un \emph{thread} alla funzione di ascolto su tale socket.
In un Super Node il thread dedicato a questa funzione è contenuto nella classe \verb|SupernodeTCPListener.java|.
Questa classe contiene un task che ha la seguente firma
\begin{verbatim}
public class OrdinarynodeTCPListener implements Runnable
\end{verbatim}
e i seguenti attributi.
\begin{lstlisting}
private NodeConfig my_conf;
private SupernodeList my_list;
private SupernodeOnFileList on_files;
private OrdinarynodeFiles my_files;
private OrdinarynodeDownloadMonitor my_dl_monitor;
private OrdinarynodeQuestionsList my_found_list;

public SupernodeTCPListener(
		NodeConfig conf,
		SupernodeList list,
		SupernodeOnFileList file_list,
		OrdinarynodeFiles sn_files,
		OrdinarynodeDownloadMonitor dl_monitor,
		OrdinarynodeQuestionsList found_list) {
	this.on_files = file_list;
	this.my_conf = conf;
	this.my_list = list;
	this.my_files = sn_files;
	this.my_dl_monitor = dl_monitor;
	this.my_found_list = found_list;
}
\end{lstlisting}
Gli attributi appena mostrati occorrono al \emph{TCPListener} per poterli poi far ereditare ai \emph{sotto-thread}. Essi poi li utilizzeranno per rispondere alle richieste che arrivano dalla rete.
Diamo uno sguardo al codice che si occupa di ricevere le richieste e far partire i \emph{sotto-thread}.

\begin{lstlisting}
//Socket che ascolta tutte le richieste
//provenienti dalla rete
ServerSocket listen_sock = null;
Socket client_socket = null;
SupernodeQueryList query_list = 
		new SupernodeQueryList();

ThreadPoolExecutor answer_pool = 
		new ThreadPoolExecutor(
		10,
		this.my_conf.getMaxConnection(),
		50000L,
		TimeUnit.MILLISECONDS,
		new LinkedBlockingQueue<Runnable>());
try {
	listen_sock = 
		new ServerSocket(this.my_conf.getPort());
} catch (IOException ex) {
}

while (true) {
	try {
		client_socket = listen_sock.accept();
		SupernodeTCPWorkingThread answer = 
				new SupernodeTCPWorkingThread(
				client_socket,
				this.my_conf,
				this.my_list,
				query_list, 
				this.on_files,
				this.my_files,
				this.my_dl_monitor,
				this.my_found_list);
		answer_pool.execute(answer);
	} catch (IOException ex) {
		//Log
	}

}
\end{lstlisting}

Il codice mostrato utilizza una tecnologia Java molto utile: \emph{Thread Pool}\ref{sec:thread_pool}.
Ogni richiesta che il client riceve sul socket TCP viene infatti delegata a un \emph{sotto-thread} che la elabora, lasciando libero il thread principale di ricevere altre richieste sul socket.



